% Okay, so the first one turned into a horrible mess.  How do I want
% to structure this one?

% Should probably present the second-order-race business as a failed
% experiment, rather than doing it in the main body of the text.

% Consider not having an explicit related work section, and instead do
% inline compare-and-contrast?

% The big thing which worries me about all of this is that, while the
% framework bits are all nice and generic, in practise you don't
% actually gain that much from them, because hard problems remain too
% hard.  Oh well.

\documentclass[10pt,a4paper,twocolumn]{article}
\usepackage{verbatim}
\usepackage{color}

\title{Speculative Lock Insertion}
\author{Steven Smith}
\begin{document}
\newcommand{\editorial}[1]{\textcolor{red}{\footnote{\textcolor{red}{#1}}}}
\newcommand{\needCite}{\editorial{need cite}}
\maketitle

\section{Missing bits}

There are a couple of bits where I say ``try X, if that doesn't work,
try Y''.  The order in which you try the various different options is
kind of important; should be more explicit about that.

\section{Introduction}

There is a well-established trend in software engineering to make
greater use of multiple threads, driven largely by the increasing
availability of multi-processor systems and multi-core processors.
While potentially paying large dividends in improved responsiveness,
throughput, and power consumption, multi-threaded programming has an
unfortunate tendency to lead to very subtle bugs.  Even worse, these
bugs are often highly dependent on the low-level details of the
hardware on which the software is being run, which can often lead to
bugs which only reproduce on some small subset of computers.  This
makes testing for these bugs extremely difficult, so they are more
likely than other kinds of bugs to remain undetected in shipped
software, and also complicates the process of developing a fix, so
once detected they remain unfixed for much longer.

In particular, desktop software is increasingly multithreaded, whereas
previously only server commonly was.  Desktop applications have
several characteristics which exacerbate the problem further:

\begin{itemize}
\item More important state is stored in volatile storage.  Server
  software is often designed to behave reasonably when power is lost,
  and this state recovery has the useful side effect of reducing the
  amount of data lost following a crash; desktop software tends not to
  have such protection, or at least not to the same degree, and so
  the risk of important data being lost is greater.

\item The workload is more heterogeneous.  The workload places on a
  server system often consists of a large number of similar requests
  carried out one after another, whereas desktop applications often
  perform more complex actions but perform them less often, with
  different users driving the application in different ways.  This means
  that different users will see different bugs.
  
  Another way of saying this is that human beings are good sources of
  randomness, and desktop applications are directly exposed to them,
  whereas servers only communicate with them at one remove.

\item The software lifecycle is shorter.  Desktop applications
  generally have more aggressive development cycles than server ones,
  which can restrict the amount of testing which is possible.

\item Desktop application developers are often less experienced in
  developing multi-threaded applications, or in some cases simply less
  able, which increases the number of bugs which are introduced.  It
  is unclear why this should be, but it does seem to be supported by
  many people's experiences\editorial{Not sure about this.}.

\item Desktop software is much less likely to come with any kind of
  useful support contract, and so it is much more difficult for
  end-users to contact the original developers in order to ask for a
  fix.
\end{itemize}

A number of techniques have been proposed for reducing the likelihood
of these multi-threading bugs, including transaction
memory\cite{Shavit1997}, message passing systems\needCite{}, or
automatic parallelization\cite{Bacon1994}.  While extremely useful,
none of these have, thus far, seen widespread deployment.  The reasons
for this are not always entirely clear, but include concerns such as
performance, a constrained problem domain, or a lack of a widely
available and widely trusted implementation.  It seems unlikely that
this situation will change in the immediate future, and there is
therefore a need for techniques which can ameliorate bugs found by
end-users in programs developed with the currently widely-used shared
memory model of concurrency.

SLI, or Speculative Lock Insertion, is one potential approach to this
problem.  It is able to fix some race-like bugs in x86 programs, given
only the program binary and a reproduction of the bug, generating a
modified binary whose behaviour is identical to that of the original
except that it no longer suffers from the nominated bug.  The fixes so
generated are safe, in the sense that any behaviour which is possible
with the fix applied would also have been possible, if sometimes
unlikely, without the fix.  Furthermore, the fixes will usually have
very low performance overhead.

In order to make this feasible, we must make some assumptions about
the nature of the bug which is to be fixed:

\begin{itemize}
\item First, we assume that it is caused by some oddity of the
  interleaving of memory accesses issued by different processors, so
  that some valid interleavings will lead to the bug reproducing and
  some will not.

\item Second, we assume that the bug is one which can be detected
  automatically.  Assertion failures and use-after-free bugs, for
  instance, would satisfy this assumption, while incorrectly rendered
  graphics would not.

\item Third, we assume that the lag between the oddity of scheduling
  and the detection of the crash is small, usually on the order of
  microseconds to milliseconds.  This is necessary to keep the
  analysis tractable.
\end{itemize}

These assumptions obviously constrain the situations in which these
techniques are applicable, but we believe that they still encompass a
reasonable selection of realistic bugs\editorial{Need the results of
  the bug tracker review here, really}.

Our approach has three basic phases:

\begin{itemize}
\item[1] Capture the bug which is to be fixed using some kind of
  deterministic replay system.
\item[2] Characterise which parts of the captured execution are
  necessary to cause the bug to reproduce.
\item[3] Create a fix which prevents the bug from reproducing.
\end{itemize}

The final stage of SLI is generating a fix.  This is, in our
prototype, a binary patch to the original binary which inserts
additional synchronisation which ensures that the bug cannot reproduce
in future executions.  To the extent that the bug has been accurately
characterised, this will fix it.  Obviously, if the characterisation
is inaccurate or incomplete then the fix will be to, but we do
guarantee that the fix will be safe (in a limited sense, defined
below).

\section{Definition of safety}

The patches which we generate have one major safety constraint: they
must not cause a correct program to become incorrect.  In the general
case, without any program-specific knowledge, this is impossible
(consider, for instance, a program which is intended to test some
aspect of a computer's memory model, and which indicates its results
by crashing in a particular way), and so the task is to find the
minimal amount of program-specific information, and hence the
strongest safety property, which allows useful fixes to be generated.

It is useful to change viewpoint at this point: rather than modifying
the program, we instead modify the semantics of the machine on which
it executes.  This is helpful because the machine is often somewhat
under-specified, and it is usually considered mad programming practise
to rely upon the unspecified parts.  This means that, for a reasonable
selection of definitions of correctness, changing unspecified aspects
of a machine's semantics cannot cause correct programs to become
incorrect.  Defining safety is then equivalent to delimiting the parts
of the machine semantics are considered to be unspecified.  In the
case of SLI, this is the time taken by individual instructions and the
way in which memory accesses issued by different threads are
interleaved.

This also provides a useful way of categorising existing
work:\editorial{This might belong in the related work section?}

\begin{itemize}
\item A number of other race-fixing systems, such as
  Isolator\cite{Ramalingam2009}, ToleRace\cite{Kirovski2007},
  Reenact\cite{Prvulovic2003a}, Atom-Aid\cite{Lucia2009}, and
  Kivati\cite{Chew2010a}, use the same instruction interleaving based
  definition as SLI.
\item Most deadlock-immunity related systems, such as
  Dimmunix\cite{Jula2008} or Gadara\cite{Wang2008}, change the time
  taken by lock acquisition operations.
\item Memory protection systems, such as DieHard\cite{Berger2006},
  Exterminator\cite{Novark2007}, or AutoPaG\cite{Lin2007}, work by
  redefining the effects of accesses to invalid memory.  Failure
  obliviousness\cite{Rinard2004} has an essentially similar model, but
  applied in a slightly more intelligent way.
\item Systems which involve backtracking, such as
  ASSURE\cite{Sidiroglou2009} or RX\cite{Qin2007}, can also be
  modelled in a similar way, if the unspecified behaviour is allowed
  to be non-causal.  These systems effectively look ahead at certain
  points and select a behaviour will avoid certain types of bugs,
  which is equivalent to saying that the instructions involved in
  those bugs influence preceding instructions so as to prevent
  themselves from ever occurring, so that the ``causes'' of program
  behaviours can occur after the program behaviours
  themselves.\editorial{Could also express this as angelic
    non-determinism; not sure which would be more understood.}
\item The semantics of the Reactive Immune System\cite{Sidiroglou2005}
  are more surprising: functions occasionally return spontaneously.
  Likewise, in ClearView\cite{Perkins} or datastructure
  repair\cite{Elkarablieh2007}, data structures and parts of data
  structures occasionally revert to previous states, or to states
  which are consistent with a previous training run.
\end{itemize}

Obviously, some of these choices are more likely to introduce new bugs
than others, and each suggests a different type of fix.

XXX -- what I really want to do is discuss what it actually means to
write a correct program in the presence of these automatic error
fixing strategies, but that doesn't really fit here.

\section{Capturing the bug}

Before SLI proper starts, the bug to be fixed must be captured using a
deterministic replay system (DRS).  This work does not attempt to
advance the state of the art in DRSes, but does depend on them in
order to be feasible, and so we discuss them briefly here.

The only requirement we place on the choice of DRS is that it must
capture an execution sufficiently precisely that the relevant fragment
of execution can later be replayed as many times as necessary at the
accuracy of individual memory accesses.  This captured execution does
not need to be precisely the same as the original crashing execution,
but the more different it is the more likely it is that SLI will fix
the wrong bug.  The most obvious way of capturing an execution, used
in our prototype, is to simply record every single memory access
issued by the program, which is effective but has extremely high
overhead.  This could be reduced by using a more intelligent recording
mechanism such as PRES\cite{Park2009} or ODR\cite{Altekar2009}, both
of which record only a few critical events and discover the rest only
when they are needed during replay.  This approach can reduce overhead
to a level where it is sensible to run with recording enabled by
default.  An extreme form of this approach is provided by
ESD\cite{Zamfir2010}, which logs nothing at all and instead attempts
to recreate the path to failure given just the state of the program at
the time of the failure.

In a slightly different context, an automatic program exerciser such
as CHESS\cite{Musuvathi2008} could be used to detect completely new
bugs, which could then be passed to SLI to be automatically
characterised and fixed.

\section{The expression language, and deriving the initial crash predictor}

Once the bug has been captured, we attempt to characterise and fix it.
It would be straightforward to ensure that the \emph{exact} program
execution does not re-occur, but doing so is unlikely to be helpful,
simply because the space of program executions is sufficiently large
that any given one would be unlikely to re-occur anyway\editorial{Need
  to define what I mean by a program execution.  I mean starting from
  a particular snapshot of memory and thread state, and then issuing
  memory accesses in a particular order}.  It is instead necessary to
generalise the execution, removing parts which are irrelevant to the
bug.  This is necessarily a somewhat informal process, because we do
not have a formal definition of the word ``bug'', and even the safest
heuristics can be undone by sufficiently peculiar correctness
specifications.  For instance, one might declare that the order of
accesses to non-overlapping memory locations is irrelevant, as that
cannot influence the program-visible state of the machine, but that
would be incorrect if the program's specification itself referred to
the order of those memory accesses (e.g. a driver performing
memory-mapped IO).  The heuristics used in our system are described
later, and generally tend to be rather
conservative\footnote{``Conservative'' in this context meaning that
  they do significantly constrain the set of possible specifications.
  They are all safe in the sense that the actual observed execution is
  always included in the generalised characterisation, although, as
  mentioned previously, this is not a terribly useful guarantee.}.

Our approach to this problem is to use the captured execution and the
program binary to derive a crash predicting expression.  This is a
function from program executions to booleans such that, if it
evaluates to true, the program will crash in the same way as it did
before.  We derive this using a restricted form of symbolic execution.

First, an initial crash predictor is produced by symbolically
executing the program, but without performing any state
forking\footnote{The implementation of this is based on LibVEX, part
  of Valgrind\cite{Nethercote2007}.}.  Instead, at every branch or
scheduling decision, we substitute in the value which was observed in
the captured execution and record the missed state fork.  Later stages
of the analysis are then able to go back and perform whichever forks
appear to be most relevant to the bug which is to be investigated.  In
this way, we are able to use the captured execution to guide the
analysis in a way which avoids the usual exponential blow up
associated with symbolic execution of complicated programs.  In our
case, we are only interested in parts of the system behaviour which
are dependent on the interleaving of memory accesses, and so the only
variables in the symbolic execution system are the values of memory
loads which might plausibly have raced\footnote{Later stages can also
  introduce explicit happens-before expressions, but they are not
  present at this stage}; everything else is treated as a constant.
This allows most thread-local computation to be constant-folded away
to nothing and thus ignored, which significantly simplifies later
analysis.

Note that we use a broad definition of a data race, which includes,
for instance, the implicit race on the lock structure when one thread
releases a lock and another subsequently acquires it.  This means that
we do not any explicit handling for synchronisation primitives, except
to the extent that blocking and unblocking threads constrains the set
of plausible schedules.

\subsection{The expression language}

The expressions themselves are generally straightforward, with a few
exceptions\editorial{There's nothing particular clever here (in fact,
  it's all pretty damn stupid), and it's taking up a lot of space.  Oh
  well.}:

\begin{description}
\item[\_|\_] (pronounced bottom) expressions are used as placeholders
  wherever the analysis cannot provide a more useful result, and could
  stand for anything at all.  In practise, \verb^_|_^ expressions are
  rare, essentially only appearing when something goes wrong, but they
  make the semantics more straightforward.

\item[onlyif] expressions, such as \verb|a onlyif b|, evaluate to
  \verb|b| if \verb|a| is non-zero and \verb^_|_^ otherwise.  

\item[load] expressions record expressions which were loaded from
  memory.  They take this form:

\begin{verbatim}
load8@1:634;2:97 422aa70:(x+10) -> 0
\end{verbatim}

  which indicates that thread \verb|1| issued a load of size \verb|8|
  at time \verb|1:634| from a constant memory address
  \verb|0x422aa70|, and that the previous store to that location came
  from thread \verb|2| at time \verb|2:97| to address \verb|(x+10)|,
  and stored the constant \verb|0|.  Timestamps, in this particular
  case, are just a pair of the thread ID and a counter of the number
  of memory accesses issued by that thread since some arbitrary
  starting point.  Note that both the load and store addresses can be
  arbitrary expressions, as can the stored value, while the timestamps
  are concrete values.

  \verb|load| expressions are, in some sense, completely redundant,
  because they evaluate to the value which was previously stored at
  that location, but the extra information is useful when refining the
  expression.

\item[lastStore] expressions capture parts of the communication
  patterns between threads.  An expression like:

\begin{verbatim}
lasStore@1:634:422a70 -> 2:97
\end{verbatim}

  evaluates to one if at time \verb|1:634| the previous store to
  address \verb|422a70| was issued by thread \verb|2| at time
  \verb|2:97|, and zero otherwise.

\item[Happens-before] expressions, written \verb|A <-< B|, evaluate to
  one if the event with timestamp \verb|A| happens before the event
  with timestamp \verb|B| and zero otherwise.

\item[RIP] expressions, named after the instruction pointer register
  on the x86-64 architecture, constrain control flow, and also serve
  to bring event timestamps into scope, as discussed above.

  For instance, the expression:

\begin{verbatim}
rip 1:{(x == y)@56f5->56fc}:X
\end{verbatim}

  expresses the constraint that \verb|x| must be equal to \verb|y| in
  order to bring accesses \verb|56f5| through to \verb|56fc| into
  scope in thread \verb|1|, and then evaluates \verb|X|.  If the
  control flow constraint is not met then the entire expression
  evaluates to \verb^_|_^.

  RIP expressions are also used to capture ``model'' executions so
  that they can be further examined.  The model execution associated
  with a particular RIP expression will be one which causes the
  control flow expression, and any enclosing expression, to be
  satisfied; beyond that, it is chosen largely arbitrarily.  The
  initial model is the execution which was initially captured, while
  the others are found by schedule exploration during the course of
  the analysis.  These model executions can then be used as a kind of
  partial oracle, providing plausible answers whenever fully general
  analysis proves impossible.  This is unsound, as we have no
  guarantee that the model execution is sufficiently representative,
  but often allows the analysis to proceed in situations where it
  would otherwise become stuck.

\end{description}

\subsection{Selecting a starting point for the replay}

The initial crash predicting expression is derived by replaying a
suffix of the log.  Selecting a correct starting place for this suffix
is important: going back too far causes leads to a very complicated
crash predicting expression, causing poor performance in the later
stages of analysis, while not going back far enough risks missing
information which is critical to producing a correct fix.  Our
approach is to start with a very short suffix (currently roughly
10ms), and then perform analysis on that.  If that produces a fix, we
terminate.  Otherwise, we look backwards through the log until we find
a store which would have been loaded by something in the crash
predictor for the current starting point, and set the starting point
to that store and retry.  If we ever find that we've had to backtrack
more than ten seconds then we terminate and try again with an enlarged
set of potentially-racing addresses.\editorial{This is currently done
  manually, and the timeout is mostly just me guessing, because the logs
  don't include it.}

\subsection{Second-order races}

This scheme relies on us being able to identify accesses which might
plausibly race.

\subsubsection{The old way}
This is more difficult than simply identifying the accesses which
\emph{did} race in some particular run, because it is possible that an
access would have raced if some \emph{later} state fork had resolved
in a different way.  To see this, consider the threads shown in figure
\ref{fig:three_threads}, and suppose that address 2 initially contains
something other than \verb|72|, and the initial schedule runs thread 1
to completion, then thread 2, and then thread 3.  That initial
schedule will show a race on address 2, but not on address 1.
Na\"ively ignoring all non-racing accesses would therefore ignore the
access to location 1 in thread 1, and so when the state fork for the
test of \verb|y| in thread 2 was performed it would be unable to
detect that a new race has been introduced.

\begin{figure*}
\begin{tabular}{lll}
\verb|x := load from address 1| &
\verb|y := load from address 2| &
\verb|store 72 to address 2| \\
&
\verb|if (y == 72)| &
\\
&
\verb|      store 7 to address 1| &
\\
\hline\\
Thread 1 & Thread 2 & Thread 3
\end{tabular}
\caption{Three threads which race}
\label{fig:three_threads}
\end{figure*}

We have two strategies for dealing with this problem:

\begin{itemize}
\item First, we try ignoring it completely, by simply assuming that
  only addresses which raced during the observed execution will suffer
  interesting races.  This is, as discussed above, unsound, but is
  often sufficient, as most real bugs do not depend on this kind of
  second-order race condition\editorial{In fact, none of the real bugs
    I've tried this on depend on it}.

\item Second, if an analysis using that heuristic fails or has to
  backtrack too far, we investigate some other possible schedules, to
  see if they exhibit any second-order effects.  We use three
  different kinds of schedules:

  \begin{itemize}
  \item Most simply, we explore every schedule which can be obtained
    from the original schedule by flipping a single existing race.
  \item Next, we try snapshoting the process periodically (currently,
    roughly every 100ms), and exploring every possible schedule
    starting from each snapshot up to some timeout (currently five
    minutes of compute time).
  \item Finally, we run the process from the initial snapshot using a
    randomly chosen snapshot (so at every race we decide independently
    which thread will win).  This is repeated a small number of times
    to achieve reasonable coverage\editorial{Not actually
      implemented}.
  \end{itemize}
  
  Exploring all of these alternative schedules is extremely expensive,
  both due to the time taken to discover the alternative addresses and
  the additional costs caused by having more racing addresses during
  the other analysis phases.  It is also of dubious value, as most
  bugs do not need it and the bugs which do are usually complicated
  enough that some later phase of the pipeline fails.\editorial{Do I
    need to mention this at all?}
\end{itemize}

\subsubsection{The new way}

We maintain a set of addresses which might race.  This initially
consists of just the address of the last race in the logfile.  If that
fails, we then consider the set which contains just the second-to-last
race, then the set which contains both the last and second-to-last
races, then just the third-to-last race, and so on.\editorial{Need
  more here.}

\subsection{How to identify memory accesses}

Our crash characterisations must often refer to events in more than
one schedule.  For instance, we might discover that a program crashes
if thread 1 access 5 happens before thread 2 access 2 or thread 1
access 1 happens after thread 2 access 3.  It is clearly impossible
for both of those happens-before relationships to occur at once, and
so the expression must refer to multiple schedules.  There is
therefore a need for some mechanism for naming memory access events
which is valid across multiple schedules, possibly with different
control flow histories.  This requires some notion of what it means
for two accesses in different executions to be equivalent.  The
appropriate choice will depend on the type of analysis to be
performed.  We use different approaches at different stages of the
analysis.

In the initial bug characterisation stage, we use a very conservative
definition of equivalence, equating two accesses iff:

\begin{itemize}
\item[a] They are issued by the same thread, and
\item[b] The issuing threads have issued the same number of preceding
  memory accesses, and
\item[c] The issuing threads followed an identical control flow path
  prior to issuing the access.
\end{itemize}

Accesses which are issued by different threads are always
non-equivalent, as are accesses by a single thread where the control
flow matches but the access numbers do not.  Otherwise, the accesses
are simply incomparable.  In this scheme, memory accesses can be
identified by a pair of thread ID and access number, provided that
steps are taken to ensure that access identifiers are never compared
across incompatible control flow histories.  In our system, we do this
using explicit control flow expressions which specify that a
sub-expression is to be interpreted with respect to a specific control
flow history, and hence effectively bring certain memory accesses into
scope, acting as a kind of binder.

During fix generation, we use a much less cautious definition, and
ignore the constraint that the issuing threads must have followed the
same control flow path, largely in order to simplify the
implementation.  This is in principle unsafe, as it means that
unrelated accesses might be conflated, but in practise does not seem
to cause any issues, because the initial analysis is usually
sufficient to constrain execution to a very small number of possible
control flows, so that the likelihood of a collision is acceptably
low.

It is also sometimes useful to be able to compare timestamps between
more different executions, allowing for some changes in the control
flow path.  In that case, we identify memory accesses by a four-tuple
containing the following items:

\begin{itemize}
\item The thread identifier.
\item The instruction pointer.
\item The number of times which that instruction has been executed by
  that thread.
\item The number of memory accesses issued by that dynamic instruction
  prior to the current one.
\end{itemize}

These timestamps still only define a particular event with respect to
some control flow history, but are far more likely to produce some
meaningful result when compared across different histories.  This
makes them useful during the initial interesting address discovery
phase, where the aim is to explore as many plausible schedules as
possible, even if that comes at the expense of some slight
inaccuracies.\editorial{Why not use this definition all the time?  The
  real answer is that I didn't think of it early enough, but there's
  also a high-horse one that it's formally equivalent to the other
  one, but more computationally expensive to work with (mostly
  maintaining per-RIP execution counters).  It's also much harder to
  sanity check that all timestamps are properly defined by RIP
  expressions with this representation.}.

This definition is still rather arbitrary, and there are many others
which would be equally valid and perhaps more useful.  For instance,
one could use the call stack augmented with the number of instructions
executed thus far in each frame, and thus eliminate the effects of
subroutines which have previously been called and have now returned,
or use per-address counters rather than per-instruction ones and hence
produce a more data-structure-oriented identifier.  We have not needed
to use these more complex approaches so far.

\section{Exploring the set of possible expressions}

For any given crash, there will usually be several possible
explanations of why it crashed, each capturing a different aspect of
the program's behaviour.  Some of these will be strictly equivalent
(e.g. \verb|x == y| versus \verb|y == x|), but, more usefully, some
will incorporate more or less of the observed schedule.  They will
therefore prohibit a greater or larger set of possible schedules,
which means that some will be easier to convert into a fix than
others.  It is therefore useful to be able to explore the various
possible expressions.  We do this using a set of linear rewrite
rules:

\begin{itemize}
\item Constant folding is applied eagerly wherever possible, along
  with some simple arithmetic simplifications.

\item \verb|BadPointer| expressions, which specify that some
  particular address is invalid, are rewritten to simply require that
  the embedded symbolic expression is equal to the embedded concrete
  address.  This amounts to assuming that the program's address space
  layout will be the same on every future run.  Ignoring
  ASLR\cite{Shacham2004}, this is a good approximation.  Even if ASLR
  is present, certain critical addresses (such as \verb|NULL| or the
  program's data segment) are likely to be unchanged, so this is
  sometimes still true then.

\item Expressions of the form \verb|load a:X;b:Y aAddr:bAddr -> val|,
  which indicates that thread \verb|a| loaded \verb|val| from symbolic
  address \verb|aAddr| at time \verb|a:X|, which was satisfied by a
  store issued by thread \verb|b| at time \verb|b:Y| using symbolic
  address \verb|bAddr|, are replaced with an expression:

\begin{verbatim}
(aAddr == bAddr &&
   (lastStore a:X b:Y aAddr))
 onlyif val
\end{verbatim}

  This amounts to saying that we only really know the value of the
  expression if the program follows the same communication graph as it
  did in the captured execution (the \verb|lastStore| expression) and
  address of the load happens to match the address of the store.
  Otherwise, nothing can be said about the value which is loaded.

\item Expressions of the form \verb|lastStore a:X b:Y addr|, which
  evaluate to true if the last store to \verb|addr| prior to
  \verb|a:X| was issued at \verb|b:Y|, are somewhat more complicated.
  They embed three main assertions:

  \begin{itemize}
  \item First, both \verb|a:X| and \verb|b:Y| must have actually
    happened.  This should already have been covered by enclosing
    \verb|rip| expressions, and so can just be assumed.
  \item Second, \verb|b:Y| must happen before \verb|a:X|.  This is
    encoded as a simple happens-before constraint \verb|b:Y <-< a:X|.
  \item Third, no other stores must intervene.  This is the most
    difficult part to check, as the other stores might not have
    happened in our model execution.  We search for these other stores
    in three places:

    \begin{itemize}
    \item First, we check for writes to the address from threads other
      than \verb|b| which happened prior to \verb|b:Y| in this
      execution but which were not ordered with respect to it.
    \item Next, we take a snapshot of the program at time \verb|a:X|,
      and then run every thread except \verb|a| forwards from there
      for a few thousand instructions\footnote{Currently 10000 memory
        accesses}, and see if they produce any racing accesses.
    \item Finally, and only if we suspect second-order races might be
      relevant, we return to the snapshot taken at \verb|a:X| and then
      perform an exhaustive enumeration of every possible schedule
      from that point forwards, up to some depth\editorial{What
        depth?}.
    \end{itemize}

    It is also necessary to check whether two stores are to the same
    address, or might have been to the same address.  This is
    essentially the aliasing problem familiar from optimising
    compilers, but applied to binaries rather than to source code.  We
    cheat and simply assume that the aliasing pattern which we
    observed in the sample execution will apply to every subsequent
    execution.  Previous iterations of this work tried much harder to
    discover whether accesses ``might have'' aliased in some other
    schedule, but we have found that this is both computationally
    extremely expensive and largely unnecessary.
  \end{itemize}
\end{itemize}

This should arguably be a proper search process, with backtracking,
rather than just a linear sequence of expressions produced by
rewriting, but that has not been necessary so far.

\section{Deriving a fix}

The refinement process produces a series of expressions which capture
different aspects of the program's execution, and hence different
causes of the crash, such that forcing any of these expressions to be
false will cause the crash to not happen.  The task is therefore to
find some change which can be made to the program which will cause the
expression to become false.

In our case, we are only interested in bugs which can be fixed by
introducing a new global mutex which is acquired in two critical
sections.  Ignoring control flow dependencies and unrolling the entire
program into simple straight-line code, a mutex which is acquired
between \verb|A| and \verb|B| and between \verb|C| and \verb|D| can be
encoded into a constraint on schedules that \verb^B <-< C || D <-< A^.
Our basic plan is then to find two ranges of dynamic execution of the
program which lead to a contradiction when encoded in this form and
combined with the crash predicting expression, map these dynamic
ranges back into static points in the program, and then attempt to
patch in suitable synchronisation at these points.

We perform a couple of simplifications to the crash predicting
expressions before beginning to look for fixes:

\begin{itemize}
\item \verb|load| expressions are replaced with the expression loaded,
  and \verb|lastStore x -> y| expressions are replaced with simple
  \verb|x <-< y| expressions.  This effectively assumes that all of
  the relevant scheduling constraints have already been captured in
  explicit happens-before relations.

\item \verb|rip X:{Y@...}:Z| expressions are replaced with
  \verb|Y onlyif Z| ones and timestamps are re-written to use the
  four-tuple form, discussed above\editorial{Timestamps \emph{should}
    be rewritten; at the moment they're not}.  This amounts to
  assuming that all relevant control flow histories are ``similar'',
  in some ill-defined sense, and so the risk of conflating unrelated
  accesses is sufficiently small even without the protection of
  explicit rip expressions.

\item \verb|Y onlyif Z| expressions are then converted to
  \verb|if Y then Z else ¬Z| expressions, effectively assuming that if
  we don't know a particular thing to be true then it must be false.
\end{itemize}

The result is that the expression contains only boolean operators,
bottom expressions, and happens-before relationships.  Given this
simplified expression, we extract every mentioned event and then
enumerate every possible critical section using those events as
boundaries (there are usually no more than a couple of dozen such
events, and so this is computationally manageable) and check each one.

Once we have a critical section, we apply two more simplification
passes before invoking the theorem prover.  We know that the original
crash predictor must be satisfiable, and so any unsatisfiability must
be in some sense related to the new critical section.  This means that
references which are unrelated to those mentioned in the critical
section are unlikely to be useful.  We take advantage of this
observation by removing them from the expression and replacing them
with bottom expressions.  An uninteresting reference is defined to be
one which isn't mentioned in the critical section and which doesn't
appear on the other side of any happens-before relation to a reference
which is mentioned in the critical section.  This is in some sense
unsafe\editorial{Do I mean unsafe, unsound, or incomplete? This whole
  paragraph needs rewriting, anyway.}, as it is possible that a
critical section might be sufficient due to chains of existing
synchronisation, and removing the intermediate references would mean
that we cannot detect this.  The net effect of this is that we are
unable to take advantage of some of the synchronisation which is
already present in the program, which can cause us to produce an
inefficient fix, or in some cases to fail to produce any fix at all.
It is possible that a more intelligent theorem prover would eliminate
the need for this simplification.

Finally, we eliminate all of the bottom expressions by means of the
following algorithm:

\begin{itemize}
\item We start in a positive context.
\item \verb^¬_|_^ is replaced by \verb^_|_^.  Otherwise, we flip the
  sense of the context and simplify \verb|x|.
\item In a positive context, \verb^x && _|_^ is replaced by \verb|x|,
  and in a negative context by \verb^_|_^.
\item In a positive context, \verb^x || _|_^ is replaced by
  \verb^_|_^, and in a negative context by \verb^x^.
\end{itemize}

If the end result is \verb^_|_^ then the expression is definitely not
a contradiction, and otherwise we need a full theorem prover.

This is supposed to be equivalent to replacing each bottom expression
with a new fresh boolean variable and then assign each variable
whichever value is least likely to be helpful to us in establishing a
contradiction.

The resulting expression is then, finally, passed to a very simple
custom theorem prover which attempts to find a contradiction.  The
details of this are completely standard and are not discussed further
here\editorial{Not sure if I can get away with that; there are
  definitely multiple standard ways of doing this, and I'm not
  convinced that my standard would be anyone else's
  standard.}\editorial{Why not use an off-the-shelf theorem prover?
  The first couple I happened to look at either didn't support
  transitive non-total relations, which is kind of important here, or
  required a lot of set-up, so I figured this would be easier.}.
  

There are usually several possible potential fixes for any given crash
expression, corresponding to different places in which the locks could
be inserted and hence different sets of executions which will be ruled
out.  All of these sections are ``correct'', within our limited
definition of the word, as they will prevent the bug from reproducing
in exactly the same way, but some will be more correct than others,
providing collateral fixes for a greater or smaller set of other ways
of reproducing it, or for different but closely-related bugs.

For instance, consider a piece of code like this:

\begin{verbatim}
1: if (global_ptr != NULL) {
2:     if (global_ptr == NULL)
3:         print_a_warning();
4:     x = global_ptr->x;
5: }
\end{verbatim}

where some other thread sets \verb|global_ptr| to \verb|NULL| without
any synchronisation\footnote{This example is taken from Thunderbird;
  the apparently-redundant inner test was produced from a function
  which was inlined by the compiler.  The example is shown in a C-like
  language for clarity; the actual analysis was, as usual, performed
  on raw machine code.}.  If our sample execution shows the
asynchronous assignment to \verb|global_ptr| occurs between the two
tests, then we will produce two critical section pairs: line 1 to line
2, and line 1 to line 4.  Both of these will be sufficient to change
the observed behaviour, but only the second, larger, one will actually
fix the bug, illustrating that selecting the smallest possible fix is
not always desirable.  There will also be, in some cases, a third
possible critical section, stretching from the initialisation of
\verb|global_ptr|, which may have occurred much earlier in the
program, to line 1.  This will also prevent the bug, as it will ensure
that the test on line 1 always sees \verb|global_ptr| to be
\verb|NULL|, so the bad code will never run, but if the initialisation
happened much earlier then it will be at high risk of introducing a
deadlock or poor performance (and will also be difficult to turn into
a patch).  This suggests that simply taking the largest critical
section is also unlikely to be a good strategy.

We use a simple heuristic to solve this problem: we select the largest
critical section set which is a superset of the smallest set and which
is less than a factor of ten larger than the smallest one (where size
is defined to be the total number of dynamic instructions across all
of the critical sections in the set).  The expectation here is that
the smallest set will fix the most proximal cause of the crash, and
hence is highly likely to be relevant, and capturing small supersets
of that will fix the most closely related bugs.  Limiting the total
size of the set helps to prevent the system from selecting
overly-broad fixes which would harm concurrency or introduce highly
likely deadlocks.  The number ten is somewhat arbitrary, but appears
to work in our (admittedly somewhat limited) evaluation.

\subsection{Implementation details of the binary patcher}

Once an appropriate dynamic critical section has been discovered, it
must be converted to a static one.  Mapping the dynamic points into
particular instructions is generally straightforward, but mapping
ranges can be more difficult if they cross control flow branches.

We simplify the problem by detecting if the start and end of a
critical section are in different functions (as delimited by
\verb|call| and \verb|ret| instructions) and, if they are,
backtracking up the call stack to their most recent common stack
frame.  Besides simplifying the binary patching problem, this has the
(sometimes) useful side-effect of expanding the critical section, and
generally moving it to boundaries which are more likely to correspond
to the original programmers' ideas about where critical section
boundaries ``should be''\editorial{Rephrase}.\editorial{\emph{Should}
  do this; currently rely on doing it by hand.}

Once we have obtained start and end points in the same function frame,
we examine the program's machine code starting from the start point,
exploring its control flow graph until we encounter the end point or
an indirect branch (the exploration process is sufficiently naive that
indirect branches cannot be sensibly predicted, and so exploration has
to stop at that point).  The CFG thus obtained is then trimmed to only
contain paths starting at the start node and ending at the end node.
The desired synchronisation can then be added to this CFG in a
straightforward way (being careful to always release any needed locks
when leaving the patch), and the result recompiled into a fragment of
position-independent machine code.  This is combined with a stub
loader and built into an ELF shared library, which can be loaded into
the target program using \verb|LD_PRELOAD| (or an equivalent runtime
mechanism using).  The stub loader is then responsible for actually
applying the patch to the program.  We use two main strategies for
doing so:

\begin{itemize}
\item The entrypoint instruction can be replaced with a direct jump to
  the patch fragment.  This is the most efficient way of gaining
  control, but is only safe if the entrypoint instruction is large
  enough to contain the jump instruction.  Otherwise, we would also
  have to change the next instruction, which is dangerous without
  performing sufficient static analysis to be confident that there are
  no undiscovered jumps to it.  Doing that kind of static analysis on
  arbitrary binary programs is challenging and we do not attempt it.

\item The debug registers\needCite{} can be used to insert a
  breakpoint at the entrypoint address, and we can then gain control
  from the resulting signal handler and transfer control to the patch
  from there.  This is simple, and always safe, but the debug
  registers can only encode four breakpoints at a time on most
  x86-type processors, which limits the number of patches which can be
  applied in this way.  It is also slower than a direct branch.

\item The entrypoint instruction can be replaced with a breakpoint
  instruction.  As the x86 breakpoint instruction is a single byte, it
  will always fit, and so this is always safe, and the number of
  breakpoint instructions is unlimited, and so this is always
  available.  Unfortunately, it is just as slow as the debug
  registers.
\end{itemize}

In practise, the second approach seems to be most useful, as most
instructions are not large enough to encode an arbitrary branch, and
few patches require more than four critical sections.

\subsection{Deadlocks}

There is a risk that introducing additional synchronisation into the
program will itself introduce a deadlock.  This is acceptable
according to the safety property given above, but is obviously
extremely undesirable in a real system.  We mitigate the issue
slightly by using a timeout on our introduced locks, so that deadlocks
at least resolve themselves in bounded time, but this can lead to
extremely poor performance or incomplete fixing of bugs.  This could
be avoided in some cases by performing static analysis to select
critical sections which are less likely to lead to deadlocks, at the
expense of potentially not being able to find any safe fix at all.

One possible approach would be to integrate with a deadlock immunity
system such as Dimmunix\cite{Jula2008}, detecting and fixing deadlocks
as they happen.  The system would then hopefully eventually converge
on a state which is both deadlock and race free.  It should also be
possible in many cases to simulate the effect which SLI lock insertion
would have on Dimmunix, and hence to determine that Dimmunix
ultimately would insert a healing lock before it needs to do so, which
would eliminate the need to reproduce the deadlock before fixing it.
Of course, the Dimmunix lock order graph is itself only an
approximation, and so this would not be guaranteed to be effective in
every case, but it seems likely that it would be sufficient in most
situations.

We have not implemented this in our current prototype.

\section{Example of the whole pipeline, applied to a simple bug}

Suppose we have a program like this (presented in C-like source code,
for clarity; the actual implementation works at the machine code
level):

\begin{verbatim}
int x;
thread1() {
    1: x = 5;
}
thread2() {
    2: local1 = x;
    3: local2 = x;
       assert(local1 == local2);
}
\end{verbatim}

Assume that the initial value of x is 7, and that we have a
reproduction in which 1 happens between 2 and 3, and that \verb|x| has
been selected as the only probably-relevant racing address.  Assuming
reasonable compiler behaviour, the initial symbolic execution pass
will produce an initial crash reason like this\footnote{This
  particular example was produced from a program compiled with gcc
  4.2}:

\begin{verbatim}
(rip 2:{(((load4@2:11;initial_state x:x -> 7) -
        (load4@2:13;1:21 x:x -> 5)) & ffffffff)
          != 0}:1)
\end{verbatim}

The main components of this expression are:

\begin{itemize}
\item The initial \verb|(rip 2:| indicates that we are concerned with
  thread 2.
\item The next component, in \verb|{}| brackets, gives a condition
  necessary for thread 2 to follow the fatal control-flow path.  Note
  that the condition is not exactly the same as that which was present
  in the original source; this is because we operate on the binary,
  and so see the effects of compiler instruction selection and
  optimisation.
\item The final component, \verb|:1|, gives the condition necessary
  for the thread to crash given that it followed the bad path.  In
  this case, once the assertion has failed, the thread is doomed, and
  so this condition is just the constant \verb|1|.
\end{itemize}

Once the crash condition has been obtained, we begin exploring
alternatives, incorporating pieces of the captured execution and
looking for variants which can be converted into a fix.  Using some
simple heuristics, we select the second load as being most likely to
be relevant, and so refine the expression to this:

\begin{verbatim}
(rip 2:{(lastStore@2:13:x -> 1:21) onlyif
  (((-5 +
     (load4@2:11;initial_state x:x -> 7)) &
    ffffffff) != 0)}:1)
\end{verbatim}

Which indicates that we only really know what happens if the load at
\verb|2:13| (statement 3) reads a value written by the store at
\verb|1:21| (statement 1), and in that case we crash if \verb|-5| plus
the load at \verb|2:11| (statement 2) is not equal to zero.  We then
use schedule exploration to find other stores which could have
satisfied the load at \verb|2:13|.  In this particular case, there
won't be any, and so the \verb|lastStore| expression is refined to a
simple happens-before expression \verb|1:21 <-< 2:13|:

\begin{verbatim}
(rip 2:{(1:21 <-< 2:13) onlyif
  (((-5 +
     (load4@2:11;initial_state x:x -> 7)) &
  ffffffff) != 0)}:1)
\end{verbatim}

The next step is to refine the remaining \verb|load| to a
\verb|lastStore| and an \verb|onlyif|:

\begin{verbatim}
(rip 2:{(1:21 <-< 2:13) onlyif
   ((lastStore@2:11:x -> initial_state)
   onlyif 1)}:1)
\end{verbatim}

This time, the condition itself becomes
\verb|(-5 + 7) & ffffffff != 0|, which can be constant-folded away to
\verb|1|; in this way, details of the program's execution beyond the
simple communication graph can be incorporated into the analysis
process.  We now refine the \verb|lastStore| expression, to try to
find what other stores could have supplied the result of the load at
\verb|2:11|.  This time, we will discover that it could also have been
satisfied by the store at \verb|1:21|, in addition to the initial
state, and so the expression will refine to:

\begin{verbatim}
(rip 2:{(1:21 <-< 2:13) onlyif
      (¬(1:21 <-< 2:11) onlyif 1)}:1)
\end{verbatim}

At this point, none of the exploration rewrite rules applies, and so
we are forced to pass this expression to fix generation.  This
simplifies to

\begin{verbatim}
(1:21 <-< 2:13 && 2:11 <-< 1:21)
\end{verbatim}

using the rules described above, and so the available timestamps are
\verb|2:11|, \verb|2:13|, and \verb|1:21|.  Suppose that we select
\verb|2:11| to \verb|2:13| as one critical section and \verb|1:21| to
\verb|1:21| as the other\footnote{In fact, our implementation will
  first try 2:13 to 2:13 against 1:21 to 1:21, but that will be almost
  immediately discovered to be ineffective and dismissed.}.  This
encodes to the expression \verb^2:13 <-< 1:21 || 1:21 <-< 2:11^, and
so the expression passed to the theorem prover is

\begin{verbatim}
(2:13 <-< 1:21 || 1:21 <-< 2:11) &&
1:21 <-< 2:13 &&
2:11 <-< 1:21
\end{verbatim}

This forms a trivial contradiction, and so the critical section is
sufficient to cause the program to behave differently in future
executions, and hence to avoid future reproductions of the same bug.

We must now instantiate these critical sections in a binary patch.  In
this case, that is very straightforward, because they cover only
straight-line code, and so we just duplicate the few necessary
instructions with the additional synchronisation.  Depending on the
details of how the compiler translated the source into machine code,
we might then gain control using either a jump instruction or the
debug registers\editorial{Always the debug registers, because the jump
  instruction bit isn't implemented.}.  This patch will, in this case,
completely fix the bug.

\section{Results when applied to some actual bugs}

XXX

Bugs to look at:

\begin{itemize}
\item At least a couple of artificial ones.
\item The glibc one.
\item The glib one.
\item The thunderbird one.
\end{itemize}

Things to include for each one:

\begin{itemize}
\item What it was
\item How many critical sections we eventually came up with, how many
  would work, whether we introduce a deadlock.
\item How long the analysis takes, and how much memory it needs.
\item Whether the binary patcher worked.
\item How much overhead the binary patch introduced (how to measure this?)
\item Whether the implemented binary patch actually fixed the bug.
\end{itemize}

Tricky part: all of my data for glibc, glib, and artificial bugs is
from a slightly older version of the algorithm, so I'll have to re-run
a lot of tests.  There are also a few manual (but mechanical) steps in
the pipeline, which makes timing information much harder to get.

\section{Related work}\editorial{This is a bit of a bestiary.  Could do with a bit more analysis.}

There have been a number of previous systems which tackle similar
problems.  Most recently, Kivati\cite{Chew2010a} attempts to fix
single-variable atomicity violations automatically by combining a
static analysis pass with some runtime support.  The result is able to
prevent many common kinds of race-like bugs with low overhead.  There
are a couple of important differences between their approach and ours:

\begin{itemize}
\item SLI is only activated once a bug has been observed, whereas
  Kivati runs at all times.  This means that it is more likely to
  ``fix'' perfectly benign races.  It also means that the fixes cannot
  easily be applied without also requiring the Kivati runtime,
  whereas, once generated, SLI fixes can stand alone without any of
  the rest of the SLI infrastructure, which may sometimes improve
  performance.

\item Kivati requires access to the program's source code during the
  initial static analysis phase, whereas SLI only requires the binary.

\item In theory, SLI can be applied to a wider class of bugs than
  Kivati, although in practise the complexity of the analysis and the
  difficulty of generating a fix for more complicated bugs means that
  this is not a particularly useful ability.
\end{itemize}

Another approach, taken by systems such as
Isolator\cite{Ramalingam2009} and ToleRace\cite{Kirovski2007},
restricts the problem domain to asymmetric races, where one thread is
correctly following a locking discipline while some other thread is
not, and seeks to ensure that the correct thread continues to be
correct despite the misbehaviour of the incorrect one.  This might,
for instance, be useful if the correct thread is controlled by an
application while the incorrect one is controlled by a library which
the application writer is unable to modify.  As with Kivati, they do
not target specific bugs.

Atom-Aid\cite{Lucia2009} is another approach to race bug
mitigation, in this case using hardware transactional memory support.
Their approach is to bundle sequences of memory accesses into
transactions according to some heuristics, effectively reducing the
number of permissible schedules and hence the scope for memory
ordering related bugs.  Provided the necessary hardware is available,
this is simple and reasonably efficient, and should also eliminate a
reasonable selection of non-trivial bugs.  The main downside of this
approach is that it requires non-standard (and presently non-existent)
hardware, which makes it less practically useful than it otherwise
would be.  There is also a philosophical argument that, as there is no
indication that a bug has been fixed, if this approach were ever to
become widely used it would lead to a kind of moral hazard, where
programmers respond to the more accommodating hardware by becoming
more sloppy, and so overall system reliability would not increase by
as much as might otherwise have been expected.\editorial{Do I really
  want this here?}

There have also been a number of attempts to automatically fix heap
management bugs, such as buffer overflows and use-after-free errors,
most recently AutoPaG\cite{Lin2007} and Exterminator\cite{Novark2007}.
These systems both take an example of a buffer overflow bug (assumed
to be deterministic) and use various analyses to determine the root
cause of the bug, eventually using this to produce a potential fix.
In that, they are remarkably similar to the system currently under
discussion; the main difference being the type of bug targeted.

All of these systems attempt to fix bugs or otherwise prevent them
from happening.  An alternative strategy is to make bugs less serious
when they do happen.  The most famous example of such a strategy is
probably failure obliviousness\cite{Rinard2004}, which waits until the
protected program makes an invalid memory reference and then attempts
to fix things up from the resulting exception handler using a number
of heuristics.  DieHard\cite{Berger2006} is conceptually similar, but
works pre-emptively rather than from a fault handler, by guessing
where memory errors are likely to occur and modifying the program's
memory map to make those errors as harmless as possible.  In this way
programs are able to continue executing in spite of the presence of
errors which would otherwise cripple them.  Failure obliviousness
cannot, however, ever fix a bug, but instead merely lessens its
effect\editorial{That's far too glib.}.  As such, these techniques can
be seen as complementary to those discussed here.

RX\cite{Qin2007} takes a third strategy.  Here, rather than
attempting to fix the bug, an attempt is made to determine which
subset of a program's functionality is bug-free, and then to restrict
the program's inputs to only exercise that functionality.  The result
is that inputs which might have triggered the bug continue to produce
incorrect output, but the damage is at least contained rather than
propagating throughout the program and potentially leading to a crash.
This is arguably safe, although not according to the definition used
in this paper, and can cover a wide variety of bugs with little
overhead.

\section{Limitations}\editorial{This veers into future work in a few places; meh.}

While the above scheme is reasonably general, it does have a number of
important limitations.  Most obviously, it can only be applied to
scheduling-related bugs which cause the program to crash quickly.
Bugs which do not lead to an obvious crash or which have a long lag
time are unlikely to be handled well by this approach.  We consider
briefly how these limitations might be ameliorated.

Bugs which do not lead to an obvious crash can be tackled in several
ways.  The most obvious approach is to simply ask the original
programmer to provide a specification of correctness, but this request
is unlikely to be well-received by many developers.  It would also be
possible to allow the user to indicate when some unspecified
undesirable thing has happened, and then just try to ensure that
future executions behave differently to the current one, but this
would probably require too many iterations to be particularly
effective.

Another approach would be to use existing error checking and debugging
tools such as Valgrind\cite{Nethercote2007} or efence\needCite{} to
convert some hidden bugs into more obvious ones, at the expense of
some performance.  Going slightly further, dynamic invariant inference
techniques such as Daikon or DIDUDE could be used to attempt to infer
how the program is ``supposed'' to behave, and hence to detect
deviations from that path, which might indicate undesirable behaviour.
The aim could then be recast from fixing bugs to making the program
behave more predictably, which is an equally valid goal and perhaps a
more general one.  Invariant inference could also be used to improve
the handling of long-lag bugs, by detecting the points at which
invariants become false; these are likely to be relevant to the
ultimate bug, which could help to direct the analysis phases to a
useful characterisation more quickly.

The handling of long-lag bugs could also be improved by employing more
static analysis.  This could, for instance, be used to determine that
a particular memory location is independent of the outcome of a
particular branch, which would help to reduce the amount of analysis
required.  This kind of analysis can be much less conservative when
performed on program source code rather than on the binary, simply
because the semantics of higher-level languages generally have
stronger semantics, and so this would be most effective when performed
with the assistance of the original developers.  Ideally, this would
take the form of a small database of properties of the program which
could be shipped with the binary, rather than requiring actual
developer interaction at bug-fixing time.  Designing this database,
and developing algorithms which can use it, is a topic for further
research.

The rather weak definition of correctness could also be regarded as a
limitation: at present, we consider any patch which would prevent the
observed execution from repeating itself to be sufficient to prevent
the bug, which clearly does not agree precisely with most people's
definition.  This can mean that we need to observe the bug several
times before producing a fix.  The critical section selection
heuristic is intended to ameliorate this, but it is not perfect.  Some
form of static analysis, as already discussed, might improve matters
here.

There is one final limitation, more of this paper than SLI itself,
which is that the evaluation is rather limited.  This makes it
difficult to judge exactly how general SLI really is, and difficult to
determine what its costs are.  This is caused largely by the
logistical difficulties associated with finding, understanding, and
then reproducing bugs in real programs, and the only real fix is for
us to continue investigating more bugs.  We hope\editorial{dread} to
do so in the future.

\section{Conclusions}

We have presented SLI, a system for automatically fixing specific
synchronisation bugs in shared-memory programs using only their
binaries, with minimal user intervention.  We have demonstrated that
it can be used to fix real-world bugs in at least some cases, and
discussed the compromises and trade-offs which are necessary in order
to produce a practically useful implementation.  While these
techniques do have a number of limitations and drawbacks, we feel that
they provide a useful basis for ongoing work to extend the set of
situations in which they are applicable.\editorial{Wibble wibble
  wibble}

\bibliographystyle{plain}
\bibliography{$HOME/mendeley.bib/library}

\end{document}
